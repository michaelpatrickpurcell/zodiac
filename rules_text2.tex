\begin{minipage}{6cm}\raggedright
%{\setmainfont[Scale=0.95]{Century Gothic-Bold}\LARGE Gameplay}\\[2.0ex]

\setmainfont{Tex Gyre Schola}

Give everyone a pencil and a player sheet. Find one black and one white six-sided dice to share.\\[1.0ex]

During the game you will draw \emph{constellations} on your sheet.
\begin{itemize}[leftmargin=*]
\vspace{-0.7ex}
\item To draw a constellation, outline a set of five adjacent squares.
\vspace{-0.7ex}
\item Each square can belong to at most one constellation.%That is, two different constellations cannot intersect.
\end{itemize}

\vspace{-0.7ex}

Each round, one player should roll the dice.  Then, you should:
\begin{enumerate}[leftmargin=*]
\vspace{-0.7ex}
\item Choose a constellation shape that corresponds to the value shown on one of the dice.% In the first round, you may choose any number you like.
\vspace{-0.7ex}
\item Draw a new constellation of the shape you chose on your sheet. You may rotate and/or flip the shape when you do so.%  Its shape must match the value that you computed in the previous step.
\end{enumerate}

%Draw the outline of your constellation on your board to indicate which five squares it contains.\\[1.125ex]

%{\setmainfont[Scale=0.95]{Century Gothic-Bold}\LARGE Scoring}\\[1.0ex]

\vspace{-0.7ex}

You will score points based on the number of star symbols in each of your constellations as follows:
\begin{center}
{
\small
\begin{tabular}{lrrrrrr} \toprule
Stars & 0 & 1 & 2 & 3 & 4 & 5 \\
Points & 0 & 1 & 3 & 6 & 10 & 15 \\\bottomrule
\end{tabular}
}
\end{center}

If you can't place a constellation, you are done.
Whoever has the highest total score when everyone is done wins the game.


%During the game, you will take turns claiming cells on a 15$\times$15 grid. To claim a cell, draw your symbol in that cell.\\[1.125ex]
%
%On the first turn, you must claim the middle cell. Thereafter, you may claim any cell that is orthogonally adjacent to a cell that has already been claimed.\\[1.125ex]
%
%To win, you must form a square by claiming the cell at each of its four corners. Such a square can be of any size and must be aligned with the grid (no diamonds).\\[1.125ex]

\end{minipage}
