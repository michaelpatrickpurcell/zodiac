\begin{minipage}{6cm}\raggedright
%{\setmainfont[Scale=0.95]{Century Gothic-Bold}\LARGE Gameplay}\\[1.0ex]

\setmainfont[Scale=0.95]{Tex Gyre Schola}

To play, first give everyone a pencil and a copy of these rules. Find two standard six-sided dice to share.\\[1.0ex]

During the game, you will partition your board into \emph{constellations}. Begin by drawing one five-square constellation on your board.\\[1.0ex]

Each round, one player should roll the dice.  Then, you should:
\begin{enumerate}[leftmargin=*]
\vspace{-0.7ex}
\item Find the number for your previous constellation's shape in the constellation table.% In the first round, you may choose any number you like.
\vspace{-0.7ex}
\item Increment or decrement that number by the difference between the two die results.% This value determines the shape of the constellation that you will add to your board this round.
\vspace{-0.7ex}
\item Find your new number in the constellation table. Draw a new constellation on your board of the corresponding shape.%  Its shape must match the value that you computed in the previous step.
\end{enumerate}

%Draw the outline of your constellation on your board to indicate which five squares it contains.\\[1.125ex]

%{\setmainfont[Scale=0.95]{Century Gothic-Bold}\LARGE Scoring}\\[1.0ex]

You will score points based on the number of stars in each of your constellations as follows:
\begin{center}
{
\small
\begin{tabular}{lrrrrr} \toprule
Stars & 1 & 2 & 3 & 4 & 5 \\
Points & \phantom{1}1 & \phantom{1}3 & \phantom{1}6 & 10 & 15 \\\bottomrule
\end{tabular}
}
\end{center}

If you cannot place a constellation, you are done.
Whoever has the highest total score wins the game.


%During the game, you will take turns claiming cells on a 15$\times$15 grid. To claim a cell, draw your symbol in that cell.\\[1.125ex]
%
%On the first turn, you must claim the middle cell. Thereafter, you may claim any cell that is orthogonally adjacent to a cell that has already been claimed.\\[1.125ex]
%
%To win, you must form a square by claiming the cell at each of its four corners. Such a square can be of any size and must be aligned with the grid (no diamonds).\\[1.125ex]

\end{minipage}
